\begin{frame}[plain]
    \begin{bee}[Summary]
        \begin{enumerate}[1.]
            \item Introduction
            \item Concepts
            \item Operations
            \item Web Platform
            \item Conclusion and Future work
        \end{enumerate}
    \end{bee} 
\end{frame}

\begin{frame}[plain]
    \begingroup
        \fontfamily{qag}\selectfont
        \Huge\color{black}\textbf{1. Introduction}\\[0.6em]
    \endgroup
\end{frame}

\begin{frame}{1. Introduction - Context}
    \begin{bee}
        \begin{enumerate}[$\bullet$]
            \item In knowledge representation, \textbf{ontologies} provide a formal structure for concepts and relationships in a domain.
            \item Essential in AI, data integration, and the \textbf{Semantic Web} for consistent data linking.
            \item Growing data makes maintenance complex and ontologies can quickly become outdated or inconsistent.
            \item Expansion relies on \textbf{reasoners}, which can be computationally expensive, especially for large-scale or real-time updates.
    \end{enumerate}
    \end{bee}
\end{frame}

\begin{frame}{1. Introduction - Main Aims}
    \begin{bee}
        \begin{enumerate}[$\bullet$]
            \item \textbf{Goal:} Develop an efficient software for ontology expansion using \textbf{SPARQL CONSTRUCT} queries.
            \item \textbf{Phase 1:} Map ontological axioms to SPARQL queries.
            \item \textbf{Phase 2:} Web app applies the queries, updates ontology dynamically, and shows results in real time.
            \item \textbf{Outcome:}  
                \begin{enumerate}[$\bullet$]  
                    \item Automates reasoning, no manual queries needed.  
                    \item Reduces reliance on external reasoners.  
                    \item Enables fast, scalable knowledge updates.  
                    \item Supports evolving information systems.  
                \end{enumerate}
        \end{enumerate}
    \end{bee}
\end{frame}

\begin{frame}[plain]
    \begingroup
        \fontfamily{qag}\selectfont
        \Huge\color{black}\textbf{2. Concepts}\\[0.6em]
    \endgroup
\end{frame}

\begin{frame}{2. Concepts - Ontologies}
    \begin{tikzpicture}[remember picture, overlay]
        \node at ([xshift=4.5cm,yshift=-5cm]current page.north west) {\includegraphics[width=7cm]{2-content/ontology-structure.png}};
    \end{tikzpicture}

    \begin{columns}
        \begin{column}{5cm}
        \end{column}
        \begin{column}{5cm}
            \begin{bee}[]
                Ontologies structure knowledge by defining concepts, relationships, and rules, enabling data standardization and human-machine understanding. They consist of classes, properties, and individuals.
            \end{bee}
        \end{column}
    \end{columns}
\end{frame}

\begin{frame}{2. Concepts - RDF and OWL}
    \begin{bee}
        \begin{enumerate}[$\bullet$]
    \item \textbf{RDF:} Standard model for representing web data using triples \texttt{<subject, predicate, object>}.
    \item \textbf{OWL:} Built on RDF, provides more expressiveness and supports reasoning with ontologies.
    \item Features:
    \begin{enumerate}[$\bullet$]
        \item \textbf{Hierarchy:} Supports subclass relationships for structured organization.
        \item \textbf{Property Characteristics:} Transitivity, symmetry, functionality, and inverse properties.
        \item \textbf{Axioms:} Enable logical reasoning and inference of new knowledge.
    \end{enumerate}
\end{enumerate}
    \end{bee}
\end{frame}

\begin{frame}{2. Concepts - RDF and OWL}
    \begin{bee}[]
        \texttt{<Jim> <hasAge> "26"} \\ 
        \texttt{<Person> rdf:type rdfs:Class.} \\
        \texttt{<Employee> rdfs:subClassOf <Person>.}\\
        \texttt{<employs> owl:inverseOf <worksFor> .} \\
    \end{bee}    
\end{frame}

\begin{frame}{2. Concepts - SPARQL}
    \begin{bee}
        \begin{enumerate}[$\bullet$]
        \item Query language and protocol for RDF data to extract, filter, and construct data from RDF graphs.  
        \item SPARQL Components:
        \begin{enumerate}[$\bullet$]
           \item \textbf{SELECT:} Specifies the variables to retrieve. It is the core component for data extraction.
            \item \textbf{CONSTRUCT:} Generates new triples based on query results, enabling data transformation.  
            \item \textbf{INSERT/DELETE:} Adds or removes triples from the RDF dataset.
            \item \textbf{WHERE:} Defines the triple pattern to match in the RDF graph.  
        \end{enumerate}
    \end{enumerate}
    \end{bee}
\end{frame}

\begin{frame}{2. Concepts - SPARQL}
    \begin{bee}
        \texttt{PREFIX ex: <http://example.org/> \\
SELECT ?manager \\
WHERE \{\\
?manager ex:manages/ex:hasAge ?age . \\
FILTER (?age > 40)\\
\}}
    \end{bee}
\end{frame}

\begin{frame}{2. Concepts - SPARQL}
    \begin{bee}
        \texttt{PREFIX ex: <http://example.org/> \\
CONSTRUCT \{\\
?person ex:aboveForty "true" . \\
\} WHERE \{\\
?person ex:hasAge ?age . \\
FILTER (?age > 40)
\}}
    \end{bee}

    \begin{bee}
        \texttt{INSERT DATA \{ \\ <subj> <prop> <obj> . \\ \}}
    \end{bee}
\end{frame}

\begin{frame}[plain]
    \begingroup
        \fontfamily{qag}\selectfont
        \Huge\color{black}\textbf{3. Operations}\\[0.6em]
    \endgroup
\end{frame}

\begin{frame}{3. Operations - Case study}
        \begin{tikzpicture}[remember picture, overlay]
        \node at ([xshift=4.5cm,yshift=-5cm]current page.north west) {\includegraphics[width=5cm]{2-content/targaryen.jpg}};
    \end{tikzpicture}

    \begin{columns}
        \begin{column}{5cm}
        \end{column}
        \begin{column}{5cm}
            \begin{bee}[]
                Simplified ontology inspired by House Targaryen from \textit{Game of Thrones}. Includes relationships for family ties, alliances, and ownership.
            \end{bee}
        \end{column}
    \end{columns}
\end{frame}

\begin{frame}{3. Operations - List}
    \begin{bee}
        \begin{enumerate}[1.]
            \item Instance Checking
            \item Hierarchy 
            \item Property Inference
            \item Transitive Property Reasoning
            \item Symmetric Property Reasoning
            \item Restrictions
            \item Equivalence Reasoning
            \item Property Chain
            \item Consistency Checking
        \end{enumerate}
    \end{bee}
\end{frame}

\begin{frame}{3. Operations - Selected}
    \begin{bee}
        \begin{enumerate}[1.]
            \item Symmetric Property Reasoning
            \item Property Chain
        \end{enumerate}
    \end{bee}
\end{frame}

\begin{frame}{3. Operations - Symmetric Property Reasoning}
    \begin{bee}
        $P \subseteq \texttt{owl:SymmetricProperty} \Rightarrow (P(x, y) \rightarrow P(y, x))$
    \end{bee}
    \begin{bee}
        \texttt{CONSTRUCT \{\\
?subj ?prop ?target .\\
\}
WHERE \{\\
?prop a owl:ObjectProperty, owl:SymmetricProperty .\\
?target ?prop ?subj .\\
FILTER NOT EXISTS \{ ?subj ?prop ?target .\}\\
\}
        }
    \end{bee}
\end{frame}

\begin{frame}{3. Operations - Symmetric Property Reasoning}
    \begin{table}[H]
\centering
\begin{tabular}{|l|}
\hline
\textbf{Existing Triples:}\\
\texttt{targaryen:isAllyOf a owl:ObjectProperty} \\
\texttt{targaryen:isAllyOf a owl:SymmetricProperty} \\
\texttt{targaryen:houseBaratheon targaryen:isAllyOf targaryen:houseTargaryen} \\
\hline
\textbf{Result Triples:}\\
\texttt{targaryen:houseTargaryen targaryen:isAllyOf targaryen:houseBaratheon} \\
\hline
\end{tabular}
\end{table}
\end{frame}

\begin{frame}{3. Operations - Property Chain}
    \begin{bee}
        $R_1 \circ R_2 \rightarrow R_3 \Rightarrow (R_1(x, y) \land R_2(y, z) \rightarrow R_3(x, z))$
    \end{bee}
    \begin{bee}
        \texttt{CONSTRUCT \{\\
?x <{superProperty}> ?z .\\
\} WHERE \{ \\
?x <{propertyChain}> ?z .\\
FILTER NOT EXISTS \{\\
?p a owl:IrreflexiveProperty .\\
FILTER(?p = <{superProperty}> \&\& ?x = ?z)\\\}\}
}
\end{bee}
\end{frame}

\begin{frame}{3. Operations - Property Chain}
    \begin{bee}
        \texttt{SELECT DISTINCT ?superProperty \\
        (GROUP\_CONCAT(CONCAT("<", STR(?subProperty), ">"); SEPARATOR="/") AS ?propertyChain)\\}
\end{bee}
\end{frame}

\begin{frame}{3. Operations - Property Chain}
    \begin{bee}
        \texttt{WHERE \{\\
?superProperty owl:propertyChainAxiom ?chainList .\\
?chainList rdf:rest*/rdf:first ?subProperty .\\
{?subProperty a owl:ObjectProperty .}\\
UNION \{\\
?listNode rdf:first ?subProperty .\\
?listNode rdf:rest rdf:nil .\\
?subProperty a owl:DatatypeProperty .\\
\}}
GROUP BY ?superProperty ?chainList
\end{bee}
\end{frame}

\begin{frame}{3. Operations - Property Chain}
\begin{table}[H]
\centering
\begin{tabular}{|l|}
\hline
\textbf{Existing Triples:}\\
\texttt{targaryen:hasSibling owl:propertyChainAxiom <Chain> } \\
\texttt{<Chain> rdf:first targaryen:hasParent} \\
\texttt{<Chain> rdf:rest*/rdf:first targaryen:hasChild} \\
\texttt{targaryen:Daenerys targaryen:hasParent targaryen:AerysII} \\
\texttt{targaryen:AerysII targaryen:hasChild targaryen:Rhaegar} \\
\hline
\textbf{Result Triples:}\\
\texttt{targaryen:Daenerys targaryen:hasSibling targaryen:Rhaegar} \\
\hline
\end{tabular}
\end{table}
\end{frame}

\begin{frame}[plain]
    \begingroup
        \fontfamily{qag}\selectfont
        \Huge\color{black}\textbf{4. Web Platform}\\[0.6em]
    \endgroup
\end{frame}

\begin{frame}{4. Web Platform - Context}
    \begin{bee}
        \begin{enumerate}[$\bullet$]
    \item Built with \texttt{Next.js} and connects to a local GraphDB instance via its REST API.
    \item Design goals: simplicity and clarity, inspired by GraphDB Workbench.
    \item Includes four pages:
    \begin{itemize}
        \item \textbf{Home}: overview and query showcase.
        \item \textbf{Inspect}: ontology exploration (classes, properties, individuals, IRI lookup).
        \item \textbf{Expand}: inference through \texttt{CONSTRUCT} queries with optional \texttt{INSERT} of derived triples.
        \item \textbf{Check}: consistency-checking queries for detecting logical issues.
    \end{itemize}
\end{enumerate}
    \end{bee}
\end{frame}

\begin{frame}{4. Web Platform - Home}
    \begin{figure}
        \centering
        \includegraphics[width=12cm]{2-content/homepage.png}
    \end{figure}
\end{frame}

\begin{frame}{4. Web Platform - Inspect}
    \begin{figure}
        \centering
        \includegraphics[width=12cm]{2-content/inspectpage.png}
    \end{figure}
\end{frame}

\begin{frame}{4. Web Platform - Expand}
    \begin{figure}
        \centering
        \includegraphics[width=12cm]{2-content/expandpage.png}
    \end{figure}
\end{frame}

\begin{frame}{4. Web Platform - Expand}
    \begin{figure}
        \centering
        \includegraphics[width=12cm]{2-content/rangeresult.png}
    \end{figure}
\end{frame}

\begin{frame}{4. Web Platform - Check}
    \begin{figure}
        \centering
        \includegraphics[width=12cm]{2-content/checkpage.png}
    \end{figure}
\end{frame}

\begin{frame}[plain]
    \begingroup
        \fontfamily{qag}\selectfont
        \Huge\color{black}\textbf{5. Conclusion and Future work}\\[0.6em]
    \endgroup
\end{frame}

\begin{frame}{5. Conclusion and Future work}
    \begin{bee}
        \begin{enumerate}[$\bullet$]
    \item Explored SPARQL as an alternative to traditional ontology reasoners.
    \item Developed SPARQL \texttt{CONSTRUCT} queries to reproduce core reasoning tasks.
    \item Built a web platform to run these queries, visualize results, and expand the ontology.
    \item Limitations: SPARQL’s expressiveness and reliance on manual GraphDB setup.
    \item Future work: increase query coverage, enhance platform features, improve portability, and benchmark performance.
\end{enumerate}

    \end{bee}
\end{frame}

\begin{frame}[plain]
    \begingroup
        \fontfamily{qag}\selectfont
        \Huge\color{black}\textbf{Thank you for your attention!}\\[0.6em]
    \endgroup
\end{frame}

\begin{frame}[plain]
\end{frame}